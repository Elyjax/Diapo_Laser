\documentclass{beamer}

\usepackage{lmodern}
\usepackage[french]{babel}
\usepackage[utf8]{inputenc}
\usepackage[T1]{fontenc}
\usepackage{makecell}
\usetheme{Warsaw}

\beamertemplatenavigationsymbolsempty
\date{}
\author{Arthur Burgada \and Pierre-Hugues Blelly}

\title{Détection des ondes gravitationnelles}
\begin{document}

\begin{frame}
	\titlepage
\end{frame}

\begin{frame}
	\frametitle{Plan de l'exposé}
	\tableofcontents
\end{frame}

\section{Les ondes gravitationnelles}
\begin{frame}
	\frametitle{Découverte théorique}
	\begin{itemize}
		\item 1916 : Relativité Générale de Einstein
		\item En linéarisant les équations un terme d'onde progressive apparaît
		\item Artefact mathématique ou réalité physique ?
	\end{itemize}
\end{frame}

\begin{frame}
\frametitle{Propriétés}
	\center\includegraphics[scale = 0.4]{Docs/tintin.png}
	\bigskip
	\begin{itemize}
		\item Analogue à une onde électromagnétique : émis lorsque un corps massique accélère
		\item Vitesse de propagation c
		\item Mais décroissance en 1/R
		\item Amplitude extrêmement faible
	\end{itemize}
\end{frame}

\begin{frame}
	\frametitle{Ordres de grandeur}
	\begin{tabular}{|m{3.5cm}|c|c|c|}
		\hline
		Source émettrice &
		Distance &
		Amplitude (m) &
		Puissance (W)\\\hline
		Cylindre d'acier de 500 tonnes tournant à 5 tours/s autours de son axe & 1 m & 
		$2 \cdot 10^{-34}$ & $10 \cdot 10^{-29}$\\\hline
		Bombre H, 1 Megatonne & 10 km & $2 \cdot 10^{-39}$ & $10^{-11}$\\\hline
		Supernova de 10 masses solaires & 10 Mpc & $10^{-21}$ & $10^{44}$\\\hline
		Coalescence de 2 trous noirs de 10 masses solaires chacun & 10 Mpc & $10^{-20}$ & $10^{50}$\\\hline
	\end{tabular}
\end{frame}


\begin{frame}
	\frametitle{Mise en évidence indirecte : Pulsar de Hulse et Taylor}
	\begin{columns}
	\begin{column}{0.5\textwidth}
		\includegraphics[scale=0.2]{Docs/pulsar_binaire.jpeg}
	\end{column}
	\begin{column}{0.5\textwidth}
	\begin{itemize}
		\item Couple de 2 étoiles dont l'une est une étoile à neutrons
		\item PSR B1913+16 découvert en 1974
	\end{itemize}
	\end{column}
	\end{columns}
\end{frame}

\begin{frame}
	\frametitle{Mise en évidence indirecte : Pulsar de Hulse et Taylor}
	\begin{columns}
	\begin{column}{0.5\textwidth}
		\includegraphics[scale=0.3]{Docs/period_shift_pulsar.png}
	\end{column}
	\begin{column}{0.5\textwidth}
		\begin{itemize}
			\item Période de 7,75 heures
			\item Diminution de la période dûe à l'émission d'ondes gravitationnelles
		\end{itemize}

	\end{column}
	\end{columns}
\end{frame}

\begin{frame}
	\frametitle{Enjeux du projet LIGO/VIRGO}
	\begin{itemize}
		\item Mise en évidence directe
		\item Précision suffisante
		\item Evaluer le taux d'expansion de l'Univers de manière indépendante de la technique utilisant la luminosité des supernovas
	\end{itemize}

\end{frame}

\section{Présentation du Michelson}
\begin{frame}
	\frametitle{Interféromètre de Michelson}
	\includegraphics[scale=0.5]{Docs/interferometre_michelson.png}
\end{frame}


\begin{frame}
	\frametitle{Principe de la détection}
	\begin{columns}
		\begin{column}{0.5\textwidth}
			\includegraphics[width=\textwidth]{Docs/detection.png}		
		\end{column}
		\begin{column}{0.5\textwidth}
			Distortion de l'espace-temps
			\\
			On détecte cette distortion grâce à un interféromètre
		\end{column}
	\end{columns}
\end{frame}

\section{Présentation des interféromètres LIGO / VIRGO}
\begin{frame}
	\frametitle{L'interféromètre VIRGO}
	\begin{columns}
		\begin{column}{0.5\textwidth}
			\small
			\begin{enumerate}[-]
				\item 2 bras de 4km de long parfaitement horizontaux (Sous vide)
				\item Un système optique  totalement isolé de l'exterieur
			\end{enumerate}
		\end{column}
		\begin{column}{0.5\textwidth}
			\includegraphics[scale=.5]{Docs/virgoview.png}
		\end{column}
	\end{columns}
	\bigskip
	3 interféromètres: VIRGO (Italie) et LIGO(Hanford(Washington) / Livinston (Louisianne))
\end{frame}

\section{LASER utilisés}
\begin{frame}
	\frametitle{Système à injection}
	\begin{columns}
		\begin{column}{0.5\textwidth}
			\includegraphics[width=\textwidth]{Docs/NPRO.png}
		\end{column}
			

		\begin{column}{0.5\textwidth}
				\begin{itemize}
					\item 2 Lasers: Un laser maître et un laser esclave
					\item On injecte un rayonnement laser dans la cavité du second laser pour faire changer son gain et modifier la fréquence d'émission du second laser.
				\end{itemize}
		\end{column}
	\end{columns}
\end{frame}

\begin{frame}
	\frametitle{Le laser de VIRGO}
	Fonctionne en deux étapes
	\begin{enumerate}[1.]
		\item Emission du laser Maître
		\item Emission du laser Esclave 
	\end{enumerate}
	\begin{table}
	\small
	\centering
	\begin{tabular}{|l|l|l|}
	\hline
		\thead{Taille du faisceau \\ ($W_0$ [mm])}& \thead{Longueur de Rayleigh \\ ($z_0$ [m])} & \thead{Divergence du \\ faisceau ($\theta_0$ [$\mu rad$])} \\
		\hline
		4.5 +/- 0.5 & 60 +/- 10 & 75 -/+ 10 \\
		\hline
	\end{tabular}
	\end{table}

\end{frame}

\begin{frame}
\frametitle{Le laser LIGO}
Fonctionnement en quatre étapes:
\begin{itemize}
	\item Emission du laser maître
	\item Emission du laser esclave
	\item Première amplification
	\item Seconde amplification
\end{itemize}
\end{frame}

\section{Cavités de Fabry-Pérot}
\begin{frame}
	\frametitle{Intérêt du dispositif}
	\begin{columns}
		\begin{column}{0.5\textwidth}
		\includegraphics[width=\textwidth]{Docs/fabry_perot.jpg}
		\end{column}
		\begin{column}{0.5\textwidth}
		Problème : 4km insuffisant pour obtenir une figure d'intérférence dûe aux ondes gravitationnelles\\
		Solution : On fait des aller-retours en utilisant des miroirs!
		\bigskip\\
		280 aller-retours : 1120km parcourus (de fait le Michelson le plus précis au monde actuellement)
		\end{column}
	\end{columns}
\end{frame}


\section{Cavités de recyclage}
\begin{frame}
	\frametitle{Cavités de Recyclage}
	\includegraphics[scale=0.3]{Docs/recycling_mirrors.png}\\
	200W en entrée => 750kW nécessaire (facteur 3750)
\end{frame}

\begin{frame}
	\frametitle{Les Miroirs}
	\begin{columns}
	
		\begin{column}{0.5\textwidth}
			\begin{itemize}
				\item Miroirs en silice
				\item Pertes très faibles ( < 2\%)
				\item Abbérations ($10^{-8} m$)
				\item Diamètre: 35 cm
			\end{itemize}
		\end{column}
		
		\begin{column}{0.5\textwidth}
			\includegraphics[width=\textwidth]{Docs/miroir.png}
		\end{column}
	\end{columns}
\end{frame}

\section{Conclusion}
\begin{frame}
	\frametitle{Conclusion}
	\begin{itemize}
		\item Formidable avancée technologique
		\item Découverte surmédiatisée
		\item De futurs résultats prometteurs
	\end{itemize}
	\center\includegraphics[scale=0.1]{Docs/ligo.jpeg}
\end{frame}

\end{document}
